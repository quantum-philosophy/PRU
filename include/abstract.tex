We investigate the utility of the state-of-the-art machine-learning techniques
in the context of fine-grained long-range prediction of the resource usage in a
computer cluster. To this end, a large data set of real-life traces is
considered. We describe in detail our workflow, starting from making the data
accessible for consumption and finishing by predicting the resource usage of
individual tasks multiple steps ahead. The experimental results indicate that
such fine-grained traces as the ones considered possess certain structures, and
these structures can be extracted by advanced machine-learning techniques and
subsequently utilized for making educated predictions. This information can be
of use to such a crucial component as the resource manager of the computer
cluster in question, allowing the manager to more intelligently orchestrate the
cluster.
