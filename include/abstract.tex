In order to facilitate the development of intelligent resource managers of
computer clusters, we investigate the utility of the state-of-the-art neural
networks for the purpose of fine-grained long-range prediction of the resource
usage in one such cluster. We consider a large data set of real-life traces and
describe in detail our workflow, starting from making the data accessible for
learning and finishing by predicting the resource usage of individual tasks
multiple steps ahead. The experimental results indicate that such fine-grained
traces as the ones considered possess a certain structure, and that this
structure can be extracted by advanced machine-learning techniques and
subsequently utilized for making informed predictions.
