In this paper, we investigate the utility of the state-of-the-art
machine-learning techniques in the context of fine-grained long-range prediction
of the resource usage in a computer cluster. To this end, a large data set of
real-life traces of resource usage is considered. We present an efficient
workflow, starting from making the data accessible for learning and finishing by
making predictions with respect to individual workload units multiple steps
ahead. The experimental results indicate that such detailed traces as the ones
considered are not deprived of a certain structure. This structure can be
extracted by advanced machine-learning techniques and subsequently used for
making educated predictions. These predictions, in turn, can be utilized by such
agents as resource managers in order to orchestrate the computer cluster in
question in a more intelligent way.
