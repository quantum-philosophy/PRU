We investigate the utility of the state-of-the-art neural networks in the
context of fine-grained long-range prediction of the resource usage in a
computer cluster. This prediction mechanism could be of help to the resource
manager of the cluster by allowing more intelligent orchestration strategies to
be implemented. We consider a large data set of real-life traces and describe in
detail our workflow, starting from making the data accessible for learning and
finishing by predicting the resource usage of individual tasks multiple steps
ahead. The experimental results indicate that such fine-grained traces as the
ones considered possess a certain structure, and that this structure can be
extracted by advanced machine-learning techniques and subsequently utilized for
making informed predictions.
