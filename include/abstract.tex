In this paper, we investigate the usage of the state-of-the-art recurrent neural
networks in the context of fine-grained long-range prediction of the resource
usage in a computer cluster. To this end, a large data set of real-life traces
is considered. We describe in detail our workflow, starting from making the data
accessible for learning and finishing by predicting the resource usage of
individual tasks multiple steps ahead. The experimental results indicate that
such fine-grained traces as the ones considered possess a certain structure, and
that this structure can be extracted by advanced machine-learning techniques and
subsequently utilized for making educated predictions.
