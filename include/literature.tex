Various machine-learning techniques have been extensively utilized for aiding
the design of computer systems. However, as it was noted in \sref{introduction},
the recent advancements in machine learning have not been sufficiently explored
yet in this context.

The work in \cite{dabbagh2015} is concerned with cloud data centers, and the
authors propose a framework for predicting the number of user requests along
with the required amount of resources. The framework relies on k-means
clustering \cite{hastie2009} for identifying different types of requests. It
then uses Wiener filters in order to estimate the workload for each identified
type.

In \cite{ismaeel2015}, the authors focus on forecasting user requests of a cloud
data center and propose a framework based on extreme learning machines, which
are feed-forward neural networks \cite{hastie2009}. Similar to
\cite{dabbagh2015}, k-means clustering is undertaken beforehand.

Aimed at predicting the \up{CPU} usage in a cloud environment, an ensemble model
\cite{hastie2009} is presented in \cite{cao2014}. It relies on multiple models
including an exponential smoothing, auto regressive, weighted nearest neighbors,
and most similar pattern model. The final predictions are obtained by combing
the predictions from these models by means of a scoring algorithm.

To conclude, only primitive architectures of neural networks have been
considered, and they have been applied to relatively simple tasks at relatively
small scales. Consequently, there is a palpable need for further exploration.
