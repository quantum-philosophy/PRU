Let us have a look at a number of studies aiding the design of computer systems
by applying machine learning.

In \cite{coskun2008}, temperature forecasting is based on an autoregressive
moving-average model \cite{hastie2009}, enabling the development of an efficient
thermal management strategy for multiprocessor systems. The work in
\cite{kumar2010} enhances runtime thermal management by providing an on-chip
temperature predictor based on feed-forward neural networks \cite{hastie2009}.
The analysis and mitigation of the impact of process variation that are
undertaken in \cite{juan2014} are facilitated by a linear regression model
\cite{hastie2009} trained on leakage measurements in order to predict peak
temperatures.

Closer to our work, the work in \cite{dabbagh2015} is concerned with cloud data
centers. The authors propose a framework for predicting the number of
virtual-machine requests along with the required amount of \up{CPU} and memory
resources. The framework relies on k-means clustering \cite{hastie2009} for
identifying different types of requests, and then it uses Wiener filters in
order to estimate the workload with respect to each identified type.

Similar to \cite{dabbagh2015}, the work in \cite{ismaeel2015} is focused on
forecasting virtual-machine requests of a cloud data center and relies on
k-means clustering as the first step. Unlike \cite{dabbagh2015}, the main
workhorse in the case of \cite{ismaeel2015} is extreme learning machines, which
are feed-forward neural networks mentioned earlier.

An ensemble model \cite{hastie2009} is presented in \cite{cao2014} targeted at
predicting the \up{CPU} usage in a cloud environment. It relies on multiple
models including an exponential smoothing, auto regressive, weighted nearest
neighbors, and most similar pattern model. The final predictions are obtained by
combing the predictions from these models by means of a scoring algorithm.

It can be seen that, in general, machine learning has been extensively utilized
for aiding the design of computer systems. However, as it is noted in
\sref{introduction}, the recent advancements in machine learning have not been
sufficiently explored yet in this context. In particular, the state-of-the-art
neural networks have not been studied only marginally: feed-forward networks are
the simplest and least powerful members of the family.

In addition, note that the predictions delivered by \cite{dabbagh2015,
ismaeel2015, cao2014} are coarse, aggregative. The corresponding techniques
treat virtual-machine requests or computational resources as a fluid and predict
the amount of this fluid that will arrive or be needed at the next time step,
which means that these techniques are not capable of characterizing individual
tasks. Such fine-grained information, however, can be of great help for the
resource manager of the computer cluster under consideration.

To conclude, only primitive architectures of neural networks have been
considered, and only aggregative resource-usage prediction has been addressed so
far. Consequently, there is a palpable need for further exploration and
development.
