\lettrine[findent=0.2em, nindent=0em]{\textbf{R}}{esource management} is of
great importance. It is an activity that, if adequately executed, enables one to
exploit optimally the full potential of the computer system at hand. However,
there is typically very little control over the production environment. In
particular, the actual workload that the system will have to process at runtime
is rarely known in advance. Therefore, efficient resource management is a
difficult task, which has to cope with such inevitable uncertainty.

Uncertainty can be mitigated by predicting the future and acting accordingly,
which is what proactive resource managers thrive on. However, accurate
prediction is not an easy task either. Modern, complex environments require
elaborate forecasting tools, which usually originate from machine learning
\cite{hastie2009}. Machine learning has recently received a great amount of due
attention due the renaissance in neural networks \cite{goodfellow2016}. They
seemingly effortlessly superseded the back-then state-of-the-art techniques for
data modeling subsequently prediction.

Neural networks are a highly promising assistant for resource management. In
fact, resource management has seen a number of applications of neural networks;
we shall touch upon some of them in \sref{literature}. The research in this
direction, however, has been limited. In particular, only primitive
architectures of neural networks have been considered, and they have been
applied to relatively simple problems. This state of affairs is unfortunate
provided that the machinery of modern neural networks has been nearly
revolutionary in other disciplines.

We feel strongly that more research should be conducted in order to investigate
the aid that the recent advancements in machine learning can give to the design
of resource managers. In this paper, we set out to conduct one such body of
research. More specifically, we study the resource usage in a computer cluster
and aim to predict this usage multiple steps ahead at the level of individual
tasks executed in the cluster, and, to this end, we intend to use recurrent
neural networks \cite{goodfellow2016}.

The remainder of the paper is organized as follows. Section~\ref{sec:literature}
provides an overview of the prior work. Our contributions are summarized in
\sref{contribution}. In \sref{problem}, the problem that we address is
formulated. Our pipeline for data processing is presented in \sref{data}, and
our predictive model in \sref{model}. The key operational aspects of our
workflow are described in \sref{operation}. The experimental results are
reported and discussed in \sref{result}. Lastly, \sref{conclusion} provides a
number of concluding remarks.
