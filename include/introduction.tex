\lettrine[findent=0.2em, nindent=0em]{\textbf{T}}{he field} of machine learning
has recently received a great amount of due attention from both inside and
outside of the field. This interest is chiefly ascribed to the renaissance in
the development of artificial neural networks \cite{goodfellow2016}.

Neural networks seemingly effortlessly superseded the back-then state-of-the-art
techniques for modeling the structure of given data and subsequently
characterizing unseen, predicting future, and generating similar data points.
Due to their flexibility and scalability, neural networks have become a strong
catalyst for the development of other disciplines such as computer vision,
speech recognition, and natural language processing.

The design of computer systems has also seen a number of applications of neural
networks; we shall touch upon some of them in \sref{literature}. The research in
this direction, however, has been limited. In particular, only primitive
architectures of neural networks have been considered, and they have been
applied to relatively simple problems. This state of affairs is unfortunate
provided that the machinery of the cutting-edge neural networks has been nearly
revolutionary in other disciplines.

We feel strongly that more research should be conducted in order to investigate
the aid that the recent advancements in machine learning can give to the design
of computer systems. In this paper, we set out to conduct one such body of
research. More specifically, we study the resource usage in a computer cluster
and aim to predict this usage multiple steps ahead at the level of individual
tasks executed in the cluster, and, to this end, we intend to use recurrent
neural networks \cite{goodfellow2016}.

The remainder of the paper is organized as follows. Section~\ref{sec:literature}
provides an overview of the prior work. Our contributions are summarized in
\sref{contribution}. In \sref{problem}, the problem that we address is
formulated. Our pipeline for data processing is presented in \sref{data}, and
our predictive model in \sref{model}. The key operational aspects of our
workflow are described in \sref{operation}. The experimental results are
reported and discussed in \sref{result}. Lastly, \sref{conclusion} provides a
number of concluding remarks.
