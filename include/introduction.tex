\lettrine[findent=0.2em, nindent=0em]{\textbf{R}}{esource management} is of
great importance. It is the activity that, if adequately executed, enables one
to exploit optimally the full potential of the computer system at hand. However,
there is typically very little control over the production environment. In
particular, the actual workload that the system will have to process at runtime
is rarely known in advance. Therefore, efficient resource management is a
difficult task, which has to cope with inherent uncertainty.

Uncertainty can be mitigated by predicting the future and acting accordingly,
which is what proactive resource managers thrive on. However, accurate and
useful prediction is not an easy task either. Modern computer systems are
elaborate, and their resource managers require elaborate forecasting mechanisms.

Forecasting traditionally falls within the scope of machine learning
\cite{hastie2009}. Machine learning has recently received a great amount of due
attention due the renaissance in neural networks \cite{goodfellow2016}, which
seemingly effortlessly superseded the back-then state-of-the-art techniques for
modeling and prediction.

We argue that modern neural networks constitute a highly promising assistant for
resource management. In fact, resource management has already seen a number of
applications of neural networks, which we shall touch upon in \sref{literature}.
However, the research in this direction has been limited. In particular, only
primitive architectures of neural networks have been considered, and they have
been applied to relatively simple problems. This state of affairs is unfortunate
provided that neural networks have been nearly revolutionary in other
disciplines. Therefore, we feel strongly that more research should be conducted
in order to investigate the aid that the recent advancements in machine learning
can give to the design of resource managers.

In this paper, we set out to conduct one such body of research. More
specifically, we study the resource usage in a large computer cluster and aim to
predict this usage multiple steps ahead at the level of individual tasks
executed in the cluster. This detailed and foresighted information can be of
help to resource managers; however, it is challenging to obtain. To this end, we
intend to use recurrent neural networks \cite{goodfellow2016}.

The remainder of the paper is organized as follows. Section~\ref{sec:literature}
provides an overview of the prior work. Our contributions are summarized in
\sref{contribution}. In \sref{problem}, the problem that we address is
formulated. Our pipeline for data processing is presented in \sref{data}, and
our predictive model in \sref{model}. The key operational aspects of our
workflow are described in \sref{operation}. The experimental results are
reported and discussed in \sref{result}. Lastly, \sref{conclusion} provides a
number of concluding remarks.
