Over the last decade, the field of machine learning has gained a lot of due
attention from both inside and outside of the field. This interest is chiefly
ascribed to the renaissance in the development of artificial neural networks.

Neural networks have superseded the prior state-of-the-art techniques for
modeling or learning the structure of a given data set and subsequently
characterizing unseen, predicting future, or generating similar data points. Due
to its flexibility and scalability, the technique has become a strong catalyst
for the development of such disciplines as computer vision, speech recognition,
and natural language processing.

Computer-system design has also been armed with a number of tools based on
neural networks. We shall touch upon them in \sref{prior-work}; for now, it
suffices to point out that only primitive architecture of neural networks have
been considered, and they have been applied to relatively simple tasks and small
scales.

We feel strongly that more research should be conducted in order to investigate
the aid that the recent advancements in machine learning can give to the design
of computer systems. In this paper, we are set out to conduct one such body of
research. To this end, we study the resource usage in a computer cluster and aim
to predict it using neural networks.

The remainder of the paper is organized as follows.
