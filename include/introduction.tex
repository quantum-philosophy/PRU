Over the last decade, the field of machine learning has received a great amount
of due attention from both inside and outside of the field. This interest is
chiefly ascribed to the renaissance in the development of artificial neural
networks.

Neural networks seemingly effortlessly superseded the back-then state-of-the-art
techniques for modeling the structure of given data and subsequently
characterizing unseen, predicting future, or generating similar data points. Due
to its flexibility and scalability, the technique has become a strong catalyst
for the development of other disciplines such as computer vision, speech
recognition, and natural language processing.

The design of computer systems has also seen some applications of neural
networks; we shall touch upon them in \sref{prior-work}. However, the research
in this direction has been rather limited. Only primitive architecture of neural
networks have been considered, and they have been applied to relatively simple
tasks at relatively small scales. This state of affairs is considered
unfortunate provided that neural networks have been nearly revolutionary in
other disciplines.

We feel strongly that more research should be conducted in order to investigate
the aid that the recent advancements in machine learning can give to the design
of computer systems. In this paper, we set out to conduct one such body of
research. To this end, we study the resource usage in a computer cluster and aim
to predict this usage using neural networks.

The remainder of the paper is organized as follows.
