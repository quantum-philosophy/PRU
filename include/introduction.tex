\lettrine[findent=0.2em, nindent=0em]{\textbf{R}}{esource management} is of
great importance. It is the activity that, if adequately executed, enables one
to exploit optimally the full potential of the computer system at hand. However,
there is typically very little control over the production environment. In
particular, the actual workload that the system will have to process at runtime
is rarely known in advance. Therefore, efficient resource management is a
difficult endeavor, which has to cope with inherent uncertainty.

Uncertainty can be mitigated by predicting the future and acting accordingly,
which is what proactive resource managers thrive on. However, accurate and
useful prediction is not an easy task either. Modern computer systems are
elaborate, and their resource managers require elaborate forecasting mechanisms.

Forecasting traditionally falls within the scope of machine learning
\cite{hastie2009}. Machine learning has recently received a great amount of due
attention due the renaissance in neural networks \cite{goodfellow2016}, which
seemingly effortlessly superseded the back-then state-of-the-art techniques for
modeling and prediction.

We posit that modern neural networks constitute a highly promising assistant for
resource management. Despite the fact that resource management has already seen
a number of applications of neural networks---which we shall touch upon in
\sref{literature}---the research in this direction has been limited. In
particular, only primitive architectures of neural networks have been
considered, and they have been applied to relatively simple problems. This state
of affairs is unfortunate provided that neural networks have been nearly
revolutionary in other disciplines. Therefore, we feel strongly that more
research should be conducted in order to investigate the aid that the recent
advancements in the field of machine learning can give to the design of resource
managers of computer systems.

In this paper, we set out to conduct one such body of research. More
specifically, we study the resource usage in a large computer cluster and aim to
predict this usage multiple steps ahead at the level of individual tasks
executed in the cluster. To this end, we intend to use recurrent neural networks
\cite{goodfellow2016}.

Consider the example given in \fref{example} where the \up{CPU} usage of a task
running on a machine in the cluster is depicted three times (the
solid-red-to-solid-blue lines). The three cases correspond to three different
time moments (the black dots) as viewed by the resource manager of the cluster.
The solid red lines represent the history of the usage, which is known to the
manager, while the solid blue lines represent the future usage, which is unknown
to the manager. The latter is what we are to predict in this work for each task
of interest and several steps ahead. In \fref{example}, our potential
predictions up to four steps ahead are depicted by a set of dashed blue lines.

Such detailed (for individual tasks) and foresighted (multiple steps ahead)
information about the future resource usage as the one depicted in
\fref{example} can be of great help to the resource manager. For instance,
knowing the future resource usage of the tasks that are currently being executed
in the cluster, the manager can more intelligently decide which of the cluster's
machines the next incoming task should be delegated to.

The remainder of the paper is organized as follows. Section~\ref{sec:literature}
provides an overview of the prior work and summarizes our contribution. In
\sref{problem}, the problem that we address is formulated. Our pipeline for data
processing is presented in \sref{data}, and our predictive model in
\sref{model}. The key operational aspects of our workflow are described in
\sref{learning}. The experimental results are reported and discussed in
\sref{result}. Lastly, \sref{conclusion} provides a number of concluding
remarks.
