We have presented our findings concerning the utility of the state-of-the-art
recurrent neural networks in the context of fine-grained long-range prediction
of the resource usage in a computer cluster. Our workflow---which starts from
making the data readily available for learning and finishes by predicting the
resource usage of individual tasks multiple time steps into the future---has
been described in detail and applied to a large data set of real resource-usage
traces of a computer cluster.

The experimental results suggest that the considered fine-grained traces possess
a certain structure, and that this structure can be extracted by advanced
machine-learning techniques and subsequently utilized for making educated
predictions. This information can be of use to such a crucial component as the
resource manager of the computer cluster in question, allowing the manager to
more intelligently orchestrate the cluster.
