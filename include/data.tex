Before we describe our data-processing pipeline, let us first give a brief
introduction to the considered data set: the Google cluster-usage traces
\cite{reiss2011}. The traces were collected over 29 days in May 2011 and
encompass more than 12 thousand machines serving requests from more than 900
users.

In the data set's parlance, a user request is a job; a job comprises one or
several tasks; and a task is a Linux program to be run on a single machine.
Each job is ascribed a unique \up{ID}, and each task is given an \up{ID} that is
unique in the scope of the corresponding job. Apart from other tables, the data
set contains the so-called resource-usage table. The table records the resource
usage of the executed tasks with the granularity of five minutes. Each record
corresponds to a specific task and a specific five-minute interval, and it
provides such measurements as the average and maximum values of the \up{CPU},
memory, and disk usage. There are more than 1.2 billion records, which
correspond to more than 24 million tasks or, equivalently, resource-usage traces
and to more than 670 thousand jobs.

The resource-usage table is provided in the form of 500 archives. Each archive
contains a single file with measurements over a certain time window. Such a
format is inconvenient and inefficient to work with, which is what we address in
this section. Consider now the three leftmost boxes in \fref{workflow}.

\subsection{Grouping} \slab{grouping}
At the first stage, the data from the 500 archives are distributed into separate
databases so that each such database contains all the data points that belong to
a particular job, resulting in as many databases as there are jobs. In order to
reduce the space requirements, only the used columns of the table are preserved.
In our experiments, these columns are the start time stamp, job \up{ID}, task
\up{ID}, and average \up{CPU} usage.

\subsection{Indexing} \slab{indexing}
At the second stage, an index of the traces is created in order to be able to
efficiently navigate the catalog of the databases created at the previous stage.
Each record in the index contains metadata about a single task, the most
important of which are the task ID and the path to the corresponding database.
We also include the length of the trace in question into the index in order to
be able to efficiently filter the traces by their lengths.

\subsection{Selection} \slab{selection}
At the last stage of our pipeline, a subset of the resource-usage traces is
selected using the index according to the needs of a particular training session
(to be discussed in \sref{result}) and then stored on disk. Concretely, the
traces are fetched from the databases and stored in the native format of the
machine-learning library utilized; we shall refer to a file in such a format as
a binary file. Instead of writing all the selected traces into one binary file,
we distribute them across several files. Such a catalog of binary files is
created for each of the three commonly considered parts \cite{hastie2009} of the
data at hand: one is for training, one for validation, and one for testing; see
also \fref{workflow}. We shall refer to these parts as $X_1$, $X_2$, and $X_3$,
respectively.

Lastly, it is common practice to standardize data before feeding them into a
machine-learning algorithm \cite{hastie2009}. In our case, it is done along with
creating the aforementioned three catalogs, which requires a separate pass over
the training set.

In conclusion, the benefit of the above pipeline is in the streamlined access to
the data during one or multiple training sessions. The binary files can be read
efficiently as many times as needed, and they can be straightforwardly
regenerated whenever the selection criteria change; note that the artifacts of
the procedures in \sref{grouping} and \sref{indexing} stay the same. The
presence of multiple binary files allows also for shuffling the training data at
the beginning of each training epoch.

Now we are ready to make use of the processed data.
