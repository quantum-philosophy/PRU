Consider a cluster of computers that is serving a stream of tasks which are
distributed across the cluster by a resource manager. Each task consumes certain
resources during its execution; examples include the \up{CPU} and memory usage
over time. The cluster has an adequate monitoring facility deployed so that the
resource usage of the tasks is at one's disposal.

The resource-usage trace of task $i$ is defined as a sequence of equally spaced
$d$-dimensional measurements in time, which we shall represent as the following
tensor of size $l_i \times d$:
\begin{equation} \elab{trace}
  x_i = \left(x_{i0}, \dots, x_{ik}, \dots, x_{i,l_i - 1}\right)
\end{equation}
where $x_{ik}$ is the measurement taken at time $t_{ik}$, and $l_i$ denotes the
length of the sequence. Such a sequence is called fine-grained data as it
contains multiple measurements over the execution of the task as opposed to
having only one aggregative measurement such as the average or maximum value.

Suppose that the current time relative to task $i$ is $t_{ik}$. This means that
$x_{i0}, \dots, x_{ik}$ are known. Given these previous values of the trace, our
goal is to estimate its next $h$ values denoted by $\hat{x}_{i,k + 1}, \dots,
\hat{x}_{i,k + h}$. Such an estimation is called a long-range prediction as it
provides multiple future values as opposed to only one. This operation is to be
performed for each active task of interest at any time moment of interest.

In order to attain the objective, we reside to learning from historical data.
Specifically, it is first assumed that there is a data set of resource-usage
traces available, and that these past traces are representative of the future
ones:
\begin{equation} \elab{traces}
  X = \{ x_i: i = 0, \dots, n - 1 \}
\end{equation}
where $n$ is the total number of traces, and $x_i$ is as in \eref{trace}. We
then apply machine learning to the data in order to construct an adequate model
and make predictions. We specifically aim at investigating the utility of the
state-of-the-art in machine learning; to this end, we use recurrent neural
networks \cite{goodfellow2016}.

Lastly, in order for learning to be possible, the fine-grained resource-usage
traces that we consider have to have a certain structure (not purely random),
which could be extracted and used for intelligent prediction. Investigating the
presence of such a structure is part of our objective in this work.
